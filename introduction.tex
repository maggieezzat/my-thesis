\chapter{Introduction}
\label{chap:intro}

Transport authorities in the public transport sector are constantly under pressure to offer high quality services, minimize resources and cost, and increase the efficiency of day-to-day operations. From a passenger point of view, it is essential to have punctual connections, real-time information and optimized travel times. This is achieved by having a control center with a human dispatcher regularly communicating with the vehicles drivers through reliable, state-of-the-art communication platforms like Trapeze's on-board systems. The on-board computer can aid the driver in accessing all the necessary information, as well as exchanging data with the control center or requesting a call to the dispatcher. 

Current advanced transportation systems make use of smart dispatching systems that assist the dispatcher by providing him/her with a work-flow to follow, resulting in a more optimized decision-making process. Although such systems supports both dispatchers and drivers and enables them to complete their tasks more efficiently and reliably, the involvement of humans will inevitably introduces errors. As such, we aim to automate the dispatcher side by replacing the human dispatcher with an \ac{AI}-dispatcher capable of understanding speech from the vehicle driver and triggering actions accordingly. Such approach cuts down costs, minimizes effort and eliminates human error. 


Natural language processing and contextual analysis techniques are making rapid progress with apparent impact on many applications and industries. Neural Networks also provide state-of-the-art solutions to many challenging problems. Speech Recognition field, in particular, is already showing very compelling results and many applications on top of it are practically deployed.


In light of the above, we define our problem and thesis objective as designing and implementing an automated dispatcher actions system that employs state-of-the-art speech recognition and contextual analysis techniques in order to be capable of understanding speech input from the driver and evaluating situations. It is thus able to issue optimum decisions and solutions, resulting in increased automation and efficiency, and decreased cost and effort.

This thesis aims at solving the problem by dividing it into two simpler sub-problems: the first is modeled as transcribing the input from the vehicle's driver using speech recognition. The second problem is concerned with analyzing the transcribed information and making decisions accordingly. Thus, our proposed system consists of two sub-systems which together form a pipeline: 

\begin{itemize}
	\item the first part of our system is an \ac{ASR} unit, with input as speech signals from the vehicle's driver. The output is a transcribed form of the given audio information.
	\item the second sub-system is a Text Classifier unit, which takes as input the output text generated by the \ac{ASR} unit. It utilizes neural network and state-of-the-art natural language understanding models in analyzing the textual data and issuing a proper action automatically.
\end{itemize}



Our Thesis outline goes as follows:

\begin{itemize}
	\item A brief review of Neural Networks (Feed Forward and Recurrent) is presented in section \ref{bg:s2}.
	\item In section \ref{bg:s3}, a literature review on Automatic Speech Recognition field including conventional and end-to-end approaches is provided.
	\item Section \ref{bg:s4} provides the principles for state-of-the-art contextual analysis techniques.
	\item An overview of the proposed system with its sub-systems is given in section \ref{meth:s1}. 
	\item In sections \ref{meth:s2}, \ref{meth:s3} and \ref{meth:s4}, we demonstrate work done on the first sub-system in our pipeline: the Automatic Speech Recognition Unit. Data collection and pre-processing is also included.
	\item Section \ref{meth:s5} sheds light on the second part of our system: the Text Classifier Unit.
	\item Detailed report and analysis of our achieved results is made available in chapter \ref{chap:results}.
	\item Conclusion and future work are discussed in chapter \ref{chap:concl}.
\end{itemize}


